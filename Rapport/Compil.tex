
\subsection*{Compilation}

Un \textit{Makefile} se trouve à la racine du projet. Il suffit donc d'exécuter la commande suivante à la racine du projet :

\begin{lstlisting}
$ make clock
\end{lstlisting}

\subsection*{Installation sur le PIC}

Concernant la connexion au PIC, il faut raccorder celui-ci sur le port 3 du routeur et votre ordinateur sur le port 2 après avoir mis l'ensemble sous tension.
Après que le PIC soit branché, exécuter les lignes suivantes dans le dossier contenant l'exécutable créé ("clock.hex").
\begin{lstlisting}
$ tftp 192.168.97.60
$ tftp> binary
$ tftp> trace
$ tftp> verbose
$ tftp> put clock.hex // executer cette ligne lorsque le voyant lumineux du port 3 est allume apres avoir prealablement pousse sur le bouton reset du PIC
$ tftp> quit
\end{lstlisting}

Le programme démarre sur le PIC automatiquement\footnote{Pour plus d'informations à ce sujet, nous vous renvoyons vers la page dédiée sur Foditic: \url{http://foditic.org/LINGI1113_13/missions/picUnixE.php?cidReq=LINGI1113_13}}.

